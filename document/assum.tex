\section{Assumptions and General Procedure of Reasoning}
\label{sec:assum}
\subsection{Assumptions}
\emph{Causality} We assume the discrete-time systems modelled in Simulink diagrams are \emph{causal} where the output at any time only depends on values of present and past inputs. Consequently, if inputs to a casual system are identical up to some time, their corresponding outputs must also be equal up to this time.

\emph{Single-rate} This mechanised work captures only single sampling rate Simulink models, which means the timestamps of all simulation steps are multiples of a base period $T$. Eventually, steps are abstracted and measured by step numbers (natural numbers $\nat$) and $T$ is removed from its timestamp. 

An \emph{algebraic loop} occurs in simulation when there exists a signal loop with only direct feedthrough blocks in the loop, such as instantaneous feedback without delay in the loop. \cite{Bostroem2016, Caspi2003, Dragomir2016} assume there are no algebraic loops in Simulink diagrams and RCRS~\cite{Preoteasa2017} identifies it as a future work. Our theoretical framework can reason about discrete-time block diagrams with algebraic loops: specifically check if there are solutions and find the solutions.

The signals in Simulink can have many data types, such as signed or unsigned integer, single float, double float, and boolean. %nine basic types, enumeration, complex, vector and matrix, fixed data type, and bus object. 
The default type for signals are \emph{double} in Simulink. This work uses real numbers in Isabelle/HOL as a universal type for all signals. Real numbers in Isabelle/HOL are modelled precisely using Cauchy sequences, which enables us to reason in the theorem prover. This is a reasonable simplification because all other types could be expressed using real numbers, such as boolean as 0 and 1. 

%Signal as a function from simulation time to a universal real type, model scalar now but could be extended to support vector and matrix
%
%Normal design (precondition is a condition): all outputs could be expressed as relation to inputs only, therefore even outputs are referred in preconditions but they are able to be converted into equal expressions with inputs only.
%
%Deterministic (if there is a solution for a feedback diagram, it must be unique.)
%
%Non-terminating
%
%Fixed number of inputs and outputs: SimBlock condition

\subsection{General Procedure of Applying Assumption-Guarantee Reasoning}
Simulink blocks are semantically mapped to designs in UTP where additionally we model assumptions of blocks to avoid unpredictable behaviour (such as a divide by zero error in the Divide block) and ensure healthiness of blocks. The general procedure of applying AG reasoning to Simulink blocks is given below.

\begin{itemize}
    \item Single blocks and atomic subsystems are translated to single designs with assumptions and guarantees, as well as block parameters. This is shown in Section~\ref{sec:trans}.
    \item Hierarchical block connections are modelled as compositions of designs ($I$) by means of sequential composition, parallel composition and feedback.
    \item Properties or Requirements of block diagrams ($S$) to be verified are modelled as designs as well.
    \item The refinement relation ($S \refinedby I$) in UTP is used to verify if a given property is satisfied by a block diagram (or a subsystem) or not. Our approach supports compositional reasoning according to monotonicity of composition operators in terms of the refinement relation. Provided two properties $S_1$ and $S_2$ are verified to hold in two blocks or subsystems $I_1$ and $I_2$ respectively, then composition of the properties is satisfied by the composition of the blocks or subsystems in terms of the same operator.
   \begin{align*}
      & \left(S_1 \refinedby I_1 \land  S_2 \refinedby I_2\right) \implies \left(S_1 \ op\ S_2 \refinedby I_1 \ op\ I_2\right) &
   \end{align*}
\end{itemize}
